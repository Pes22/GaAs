\chapter{دراسة خلية شمسية ذات وصلة متعددة بواسطة برنامج $ solcore  $: } % Main chapter title

\label{Chapter3} % Change X to a consecutive number; for referencing this chapter elsewhere, use \ref{ChapterX}

%----------------------------------------------------------------------------------------
%	SECTION 1
%----------------------------------------------------------------------------------------



\section{مقدمة}
في هذه المذكرة سنقوم بدراسة خلية شمسية ذات وصلة متعددة بواسطة برنامج $ solcore $، لقد قمنا باختيار خلية شمسية ذات وصلة متعددة لأجل زيادة امتصاص الطيف الشمسي الساقط وبالتالي زيادة عدد الالكترونات التي يتم تحريرها إلى الدارة الخارجة بواسطة الفوتونات، وهذا يؤدي بنا إلى خلايا شمسية ذات كفاءة أعلى. 

\section{نموذج الخلية الشمسية المدروس:}
في هذا النموذج سنقوم بانشاء خلية شمسية ذات وصلة مزدوجة من مادتي $ GaAs $ و $ GaInP $ حيث الشبيكة البلورية مرتبطة بمركب $ GaAs $، ( تحتوي الخلية السفلية على 30 بئراً كمومية متوازنة متوترة $ (QW) $ ، مصنوعة من $ GaAsP $ / $ InGaAs $.)، يوجد تقاطع نفقي بين الخلايا الفرعية، توجد طبقة عاكسة على الجهة الأمامية مصنوعة من $ MgF-ZnS $.
\\
يقوم $ solcore $ بحساب معامل الإمتصاص للبئر الكمي، الكفاءة الكمية للخلية الشمسية، وكذلك خصائص الخلية الشمسية (الكفاءة $ Eff $، تيار الدارة المستقصرة $ I_{sc} $، جهد الدارة المفتوحة $ V_{oc} $، معامل الملأ $ FF $) بدلالة شدة الإضاءة.
\section{المواد المستعملة:}
\subsection{$ Gallium Arsenide (GaAs) $}
زرنيخيد الغاليوم $ (GaAs) $ عبارة عن شبه موصل ذو فجوة نطاق مباشر ،مكون من عناصر العمو III-V من الجدول الدوري بهيكل بلوري من $ zinc blende $. يستخدم زرنيخيد الغاليوم في تصنيع الأجهزة مثل الدوائر المتكاملة بتردد الميكروويف ، الدوائر المتكاملة بالميكروويف المتجانسة ، الصمامات الثنائية الباعثة للضوء بالأشعة تحت الحمراء ، الصمامات الثنائية الليزرية ، الخلايا الشمسية والنوافذ البصرية.غالبًا ما يستخدم $ GaAs  $ كمادة ركيزة للنمو فوق المحور لأشباه الموصلات $ III-V $ الأخرى ، بما في ذلك زرنيخيد الغاليوم الإنديوم وزرنيخيد الغاليوم الألومنيوم وغيرها.

\subsection{$Gallium Indium Phosphide (GaInP) $}
فوسفيد غاليوم الإنديوم، هو شبه ناقل يتكون من عناصر العمود الثالث والخامس من الجدول الدوري، بنيته البلورية عبارة عن $ zinc blende $ بثابت شبيكة $	5.8687-0.4182x A $ ، يتم استخدامه بشكل أساسي في هياكل $ HEMT $ و $ HBT $ ، و أيضًا لتصنيع الخلايا الشمسية عالية الكفاءة المستخدمة في تطبيقات الفضاء ، وبالاقتران مع الألومنيوم (سبيكة $ AlGaInP $) لصنع مصابيح $ LED $ عالية السطوع باللون البرتقالي والأحمر والبرتقالي والأصفر والأخضر الألوان. تستخدم بعض أجهزة أشباه الموصلات مثل $ EFluor Nanocrystal $ $ InGaP $ كجسيم أساسي. 


\subsection{طريقة العمل:}

	
نحتاج أولاً إلى إنشاء هيكل الخلايا الشمسية. حيث انها مصنوعة من عدة قطع: البئر الكمي $ QW $،  وصلتين، تقاطع نفقي و طبقة عاكسة $ ARC $.
\begin{itemize}
	\item 
نبدأ بإنشاء مواد البئر الكمي، عرض البئر $ 7nm $ مع طبقات بينية من $ GaAs $ بسمك $ 2nm $ على كل جانب وحاجز من $ GaAsP $ بسمك $ 10nm $. تحتوي الخلية النهائية على $ 30 $ من هذه $ QWs $. هذه هي المواد التي تصنع $ QW $ وتنشئ "$ QWunit" $. سنستخدم هذه البنية فقط لنكون قادرين على حل $ QWs $ بعد ذلك نحولها إلى سلسلة من الطبقات ذات الخصائص الفعالة التي يفهمها  $ PDD solver $. استدعاء "$ QWunit" $ له عدة مدخلات ، بما في ذلك درجة الحرارة ، وعدد مرات تكرار $ QWs $ وبنية الطبقات. ثم نستدعي "$ GetEffectiveQW" $ التي تستخدم الأدوات المساعدة داخل وحدة ميكانيكا الكم لحساب بنية النطاق لـ $ QWs $ ، ومعامل الامتصاص الخاص بها ، وأخيراً ، ستحسب فجوة النطاق الفعالة ، وكثافة الحالات ، وما إلى ذلك التي سيستخدمها محلل $ PDD $ . على الرغم من أن الجهاز سيحتوي على 30 بئراً كمياً ، إلا أن وحدة واحدة فقط (الوحدة المشار إليها في الطبقات) سيتم تصميمها على أنها $ QW $ معزولة. (بينما يمكن لـ $ Solcore $ حل معادلة شرودنجر في بنية بأي عدد من الطبقات ، يمكن لآلة حاسبة الامتصاص لـ $ QWs $ التعامل بشكل صحيح فقط مع $ QWs $ الفردية. هذا هو سبب نمذجة $ 1 QW $ فقط على الرغم من وجود $ 30 $ في الهيكل.)
	\item 
ثم بعد ذلك علينا تعريف الوصلات (التقاطعات)، من أجل حساب خصائص تقاطع شمسي باستخدام أداة حل $ PDD $ ، نحتاج إلى إعطاء كل الطبقات والمواد التي تتكون منها الوصلات ، بالطريقة نفسها التي فعلناها مع $ QWs $. نعرف مواد الوصلة السفلية(النافذة السفلية  $ GaInP $، ثم $ GaAs $ نوع $ n $ ثم $ GaAs $ نوع $ p $ ، ثم طبقة مجال السطح الخلفي $ GaInP $). وبنفس الطريقة نهرف مواد الوصلة العلوية (النافذة العلوية  $ AlInP $، ثم $ GaInP $ نوع $ n $ ثم $ GaInP $ نوع $ p $ ، ثم طبقة مجال السطح الخلفي $ AlInP $)
	\item 
ثم بعد ذلك يتم تحديد تقاطع النفق الوحيد لهذه الخلية الشمسية وفقًا للنموذج المعياري وسنفترض أنه مصنوع من طبقات $ GaInP $ ، يبلغ سمكها الإجمالي $ 40 $ نانومتر ، والتي ستمنع جزءًا من الضوء الذي يصل إلى التقاطع السفلي. نظرًا لأن الوصلة العلوية مصنوعة أيضًا من $ GaInP $ ، يتم بالفعل امتصاص معظم الضوء وبالتالي لا توجد خسارة مهمة للغاية. سنستخدم تيار ذروة منخفض نسبيًا لإظهار تأثير انهيار تقاطع النفق عند العمل بتركيز عالٍ.
	\item 
كذلك نقوم بتعريف الطبقة العاكسة، حيث أن $ ARC $ ستقلل من انعكاس السطح الأمامي ، وبالتالي يزيد التيار الضوئي للخلية الشمسية.تم استخدام طلاء بسيط مزدوج الطبقة مصنوع من $ MgF_2 $ و $ ZnS $. كلتا المادتين متاحتان في قاعدة بيانات $ SOPRA $ للثوابت البصرية.

\end{itemize}
وبهذا نكون قد عرفنا جميع المواد والهياكل، نحتاج فقط إلى تجميع كل شيء معًا ، بما في ذلك طبقة النافذة الأمامية وطبقة $ BSF $ في التقاطع العلوي الذي تركناه بالخارج.





