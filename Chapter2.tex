
\chapter{برنامج $ solcore $}

\label{Chapter2} 
%----------------------------------------------------------------------------------------

% Define some commands to keep the formatting separated from the content 
%----------------------------------------------------------------------------------------




\section*{نظرة عامة عن برنامج $ solcore $}

\subsection{ما هو $ solcore $}
بدأ $ solcore $ كمجموعة معيارية من الأدوات، كتبت (أغلبها) بلغة البرمجة $ python $، لمعالجة بعض المهام التي كان علينا حلها في الكثير من الأحيان،كإنشاء منحنيات $ I(V) $ في الظلام أو عند إضمحلال الضوء. ومع ذلك ، فقد تطورت بمرور الوقت باعتبارها حلاً كاملاً لأشباه الموصلات قادر على نمذجة الخصائص الضوئية والكهربائية لمجموعة واسعة من الخلايا الشمسية ، من أجهزة الآبار الكمومية إلى الخلايا الشمسية متعددة الوصلات.
\subsection{لغة البرمجة python:}





\subsection{تحميل و تثبيت $ solcore $ :}
يمكنك تجربة $ Solcore $ دون تثبيت أي شيء على جهاز الكمبيوتر الخاص بك باستخدام الخدمة عبر الإنترنت $ MyBinder.org $. 
\\
كما ذكرنا من قبل أنه تمت كتابة أغلبية  $ Solcore $ بلغة $ Python $ ، لكن تمت كتابة الأداة المستعملة لحل معادلة بواسن  $ Poisson-Drift-diffusion (PDD) $ بلغة $ Fortran  $ لجعلها أكثر كفاءة ، ويستخدم محلل $ RCWA $ (تحليل الموجات المزدوجة الصارم) الحزمة $ S4 $ المكتوبة بلغة $ C ++ $. ولتثبيته على الكمبيوتر يمكن أن نستعين بالإرشادات الموجودة على الموقع الرسمي ل $ solcore $ . نحتاج أولا لتثبيت $ python  $ بنسخة 3.7 أو أعلى  و $ pip $ التي هي عادة جزء من تثبيت $ python  $ لكن يمكن تثبيتها بكتابة الأمر $ pip install -U pip setuptools wheel $ على $ terminal $، كذلك نحتاج إلى $ setuptools $ . الآن تثبيت $ solcore $ سيكون سهلا. حيث يجب علينا أن نكتب الأمر $ pip install solcore $ على $ terminal $، سيؤدي هذا إلى تنزيل $ Solcore $ من مستودع $ Pypi  $ وتثبيت الحزمة داخل مجموعة حزم $ Python $ ، لكنه لن يقوم بتثبيت أداة حل $ PDD  $، والتي تحتاج إلى مترجم فورتران المناسب لها. بافتراض أن لديك مترجم $ Fortran  $مثبتًا ومهيئًا بشكل صحيح ، يمكنك تثبيت $ Solcore $ باستخدام أداة حل $ PDD $ عن طريق كتابة الأمرين $ pip install solcore $ ثم $ pip install --no-deps --force-reinstall --install-option="--with_pdd" solcore $  على $ terminal $. و هذا كل شيء! يجب أن يكون $ Solcore $ متاحًا للاستخدام كما هو الحال مع أي حزمة $ Python $.
ملاحظة: إذا واجهتك أي مشاكل يمكن الرجوع إلى الصفحة الرئيسية ل $ Solcore $ حيث توجد طريقة مختلفة لتثبيته. 
\subsection{الهياكل و الأصناف الداعمة: }
$ Solcore $ وبكل تأكيد يتعلق بالفيزياء، ومع ذلك فهو يحتاج إلى الكثير من الأدوات الداعمة التي تعمل معا بشكل سلس أو تدعم إنشاء. وسنتطرق إلى بعض تفاصيل هذه الأدوات:
\subsubsection{الهيكلة:}
يقوم $ Solcore $ بحساب الخواص الضوئية والكهربائية للخلايا الشمسية، أي مجموعة معينة من المواد مرصوصة على شكل طبقات لتخدم أغراضا محددة. يحتوي النموذج الهيكلي على وحدات البناء الأساسية التي تسمح بإنشاء هيكل الخلية الشمسية وحساب خصائصها.
بعض الأصناف التي يحتويها $ Solcore $ 
\begin{itemize}
	\item $ class solcore.structure.Structure(*args, **kwargs) $
	\\ صنف الهيكل: فئة فرعية من القوائم التي تخزن معلومات عينة تتكون من عدة طبقات.
	\item $ class solcore.structure.Layer(width, material, role=None, geometry=None, **kwargs) $
	\\وهي فئة لتخزين المعلومات حول طبقات المواد ، مثل السُمك والتركيب. أي أنها لبنة بناء "الهياكل"
	\item $ class solcore.structure.Junction(*args, **kwargs) $
	\\وهي فئة تجمع الطبقات التي تشكل تقاطعًا (وصلة). في الأساس ،هي مجرد قائمة بسمات يمكننا تحديثها.
	\item $ class solcore.structure.TunnelJunction(*args, **kwargs) $
	\\وهي فئة تحتوي على الحد الأدنى من التعريفات لتقاطع نفق ،بمعنى مقاومة على التسلسل في هذه الحالة.
	\item $ solcore.structure.InLineComposition(layer) $
	\\يمكن القول عنها أنها تدبيرة (حيلة) لإستخدام حاسبة $ Adachi-alfa $ لإثبات أن التركيب عبارة عن سلسلة واحدة.
	\item $ solcore.structure.SolcoreMaterialToStr(material_input) $
	\\هي فئة تقوم بترجمة المواد $ (materials) $ في solcore إلى شيء (معلومات)يمكن تخزينه بسهولة في ملف وقراءته.
	\item $ solcore.structure.ToSolcoreMaterial(comp, T, execute=False, **kwargs) $
	\\هذه الفئة توفر المواد المكونة للسلسة من $ solcore $ . يمكن أن نكون النتيجة عبارة عن السلسلة مع الأمر أو مادة $ solcore $ نفسها.
	\item $ solcore.structure.ToLayer(width, material, role) $
	\\هذه الفئة تقوم بانشاء طبقة بناءً على سلاسل تحتوي على العرض والمادة والدور.
	\item $ solcore.structure.ToStructure(device) $
	\\هذه الفئة تقوم بترجمة أو تحويل موضوع الجهاز إلى قائمة بنية $ solcore $ التي يمكن استخدامها في المكونات الأخرى. في الوقت الحالي تترجم فقط التكوين. كذلك تعمل فقط بالنسبة للأجهزة غير المتداخلة (بئر كمي مثلا، لكن لا تعمل على جهاز يحتوي على بئر كمي)
	\item $ class solcore.state.State $
	\\تحدد هذه الفئة الطريقة لتوسيع سمات كائن ما ، وعادة ما يتم إصلاحها أثناء تعريف الفئة. في هذه الحالة ، تكون الفئة مجرد قاموس - نوع خاص منه - ويتم توسيع السمات بإضافة مفاتيح جديدة إليها.
\end{itemize} 
\subsubsection{الخلايا الشمسية}
أخيرًا ، يتم تضمين اللبنة الأساسية للخلايا الشمسية ذات المستوى الأعلى في وحدة الخلايا الشمسية.
$ solcore.solar_cell.default_GaAs(T) $
$ class solcore.solar_cell.SolarCell(layers=None, T=298, cell_area=1, reflectivity=None, shading=0, substrate=None, incidence=None, R_series=0, **kwargs) $
هذه الفئة متطابقة تقريبًا مع فئة الهيكل الأساسية في Solcore (إنها فئة فرعية منها ، في الواقع) ولكنها تنفذ بعض قيم المعلمات الافتراضية وتتحكم في أنواع الطبقات. يجب أن يعمل في أي مكان يعمل فيه موضوع الهيكل.
ويحتوي علة كلاسات مختلفة وهي:
\\$ sort_layer_type(layer, i) $
\\يفرز الطبقة في فئات مختلفة ، اعتمادًا على نوعها ، ويحتفظ بالسجلات في فهارس هذا النوع من الطبقات.
\\$ append(new_layer, layer_label=None, repeats=1) $
\\يقوم بإلحاق طبقة بالهيكل عددًا معينًا من المرات.
\\$ append_multiple(layers, layer_labels=None, repeats=1) $
\\يقوم بإلحاق طبقات متعددة بالهيكل لعدد معين من المرات.
\\$ update_junction(junction, **kwargs) $
\\يضيف أو يحدّث السمات - وليس الطبقات - للتقاطع(الوصلات).
\subsubsection{متتبع العلوم:}
$ Solcore $ هو عمل أصلي ، ولكن المعادلات التي ينفذها والبيانات التي يستخدمها تم نشرها في كثير من الأحيان. يسمح لك متتبع العلوم بتتبع تلك المراجع والتحقق من مصدرها وافتراضاته بنفسك.
ولديه كلاسات خاصة به. هناك بعض الكلاسات الإضافيةيمكن الإطلاع عليها من الموقع الرسمي ل$ Solcore $
\subsection{المواد والوحدات: }
هذه هي الوحدات النمطية التي تتعامل مع خصائص المواد والوحدات. جنبا إلى جنب مع وحدات الهيكل ، فإنها تشكل العمود الفقري لـ $ Solcore $.
\subsubsection{نظام المواد:}
تحتوي قاعدة بيانات المعلمات على الخصائص الأساسية للعديد من مواد أشباه الموصلات ، بما في ذلك السيليكون والجرمانيوم والعديد من السبائك الثنائية والثالثة أشباه الموصلات $ III-V $. من بين المعلمات الأخرى ، يتضمن فجوة نطاق الطاقة ، والكتل الفعالة للإلكترون والثقوب ، وثوابت الشبكة والثوابت المرنة.

\\هناك طريقتان لاسترجاع المعلمات من قاعدة البيانات. الأول يتكون ببساطة من الحصول على البيانات باستخدام وظيفة $ get-parameter $ مع المدخلات المطلوبة، تستخدم هذه الطريقة البيانات الموجودة فقط.أما الطريقة الثانية تتمثل في إنشاء كائن مادي يحتوي على جميع الخصائص الموجودة في قاعدة البيانات لتلك المادة ، بالإضافة إلى تلك المدرجة كمدخلات ، والتي ستتجاوز قيمة معلمة قاعدة البيانات ، إن وجدت.
\\
الآن ، أي معلمة - بما في ذلك المعلمات المخصصة - هي سمات يمكن الوصول إليها بسهولة واستخدامها في أي مكان في البرنامج.
\\
توضح الصورة أدناه الخريطة الثابتة ذات فجوة الحزمة المعروفة مقابل الخريطة الثابتة لجميع مواد أشباه الموصلات والسبائك (المركبات الثلاثية فقط) المطبقة حاليًا في $ Solcore $. ومع ذلك ، يمكن استخدام أي مادة أخرى في جميع وظائف $ Solcore $ ، طالما يتم توفير معلمات الإدخال الضرورية. يمكن القيام بذلك عن طريق تجاوز جميع خصائص مادة موجودة أثناء الإنشاء على النحو الوارد أعلاه ، أو إضافتها كمواد خارجية في ملفات التكوين.
********************figure1*****************

\subsubsection{أصناف المواد:}
تتحقق هذه الوظيفة مما إذا كانت المادة المطلوبة موجودة وتقوم بإنشاء فئة تحتوي على خصائصها ، بافتراض أن المادة غير موجودة في قاعدة البيانات حتى الآن.
ستعمل هذه الفئة كفئة أساسية لجميع المواد المشتقة بناءً على تلك المادة المحددة.
المواد المشتقة بناءً على مادة محددة هي أمثلة على فئة المواد المحددة.
يتم إنشاء المادة من المعلمات في نظام المعلمات والبيانات $ n $ و $ k $ إذا كانت متوفرة. إذا كانت البيانات $ n $ و $ k $ غير موجودة - على الإطلاق أو لهذا التكوين - فإن $ n = 1 $ و $ k = 0 $ في جميع الأطوال الموجية. ضع في اعتبارك أن بيانات $ n $ و k المتاحة صالحة فقط في درجة حرارة الغرفة.
\subsubsection{وحدات الحركية:}

تسمح هذه الوحدة بحساب تنقل حاملات الشحنة بناءً على تركيبة المواد ودرجة الحرارة$  (T> 100K) $ وتركيز الشوائب. أي أنه تطبيق لنموذج التنقل بواسطة $ Sotoodeh et al $. تستخدم فئة المواد المدمجة في $ solcore $ هذه الوحدة داخليًا للحصول على تعبئة المواد التي يتم تنفيذها من أجلها.
يتم تضمين معلمات المواد المستخدمة في النموذج في ملف $ mobility-parameters.json $.

\subsubsection{إضافة مواد جديدة إلى $ solcore $:}
يمكن إضافة مواد جديدة إلى $ solcore $ عن طريق تنزيل  واستخدام قاعدة البيانات من $ refractiveindex.info $ والتي توفر بيانات $ n $ و $ k $ ومعلمات أخرى لإنشاء مادة جديدة.
من أجل التحكم في مكان حفظ المواد المخصصة، نحتاج إلى إخبار $ Solcore $ بمكان إنشاء المواد الجديدة والبحث عنها عن طريق إضافة بعض الإدخالات إلى ملف تكوين المستخدم الخاص بنا (تلقائيا، مجلد مخفي يسمى .$ solcore-config.txt $ في الدليل الرئيسي الخاص بنا):

المسار الذي سيتم تنزيل قاعدة بيانات $ refractiveindex.info $ إليه يتم تعيينه ضمن $ (Other) $ مع الإشارة $ nk $.
يتم تعيين المسار حيث سيتم حفظ بيانات $ n $ و $ k $ ضمن $ (Other) $ مع وضع علامة $ custom_mats $.
يتم تعيين المسار حيث سيتم إنشاء الملف الذي يحتوي على معلمات المواد المخصصة ضمن [المعلمات] مع علامة مخصصة.

يحدد مقتطف الشفرة التالي موقع كل منها داخل مجلد يسمى $ Solcore $ ، وهو دليل فرعي لمجلدك الرئيسي (يمكنك أيضًا إضافة المسارات الصحيحة يدويًا إلى ملف التكوين).


\subsection{الحلول الكمية:}
إن بنية النطاق الإلكتروني لمواد أشباه الموصلات مسؤولة عن امتصاص الضوء وخصائص الانبعاث بالإضافة إلى العديد من خصائص النقل الخاصة بها ، والتي تعتمد في النهاية على الكتل الفعالة للناقلات. هذه الخصائص ليست متأصلة في المادة ، ولكنها تعتمد على عوامل خارجية أيضًا ، وأبرزها الإجهاد والحصر الكمي. حتى أولئك الذين لديهم نفس ثابت الشبكة قد يتعرضون للضغط بسبب تأثيرات أخرى مثل وجود الشوائب أو إذا تم استخدامها في درجات حرارة مختلفة لها معاملات تمدد حراري متباينة. يحدث الحبس الكمي ، بدوره ، عندما يتم تقليل حجم مادة أشباه الموصلات في بُعد واحد أو أكثر إلى بضعة نانومترات. نظرًا لأن المواد المجهدة ذات الحصر الكمي ، يجب توخي الحذر بشكل خاص للحصول على مجموعة معقولة من المعلمات لهياكل $ QW  $، بما في ذلك حواف النطاق مع مستويات الطاقة المحصورة والكتل الفعالة ومعامل الامتصاص.
يتضمن نهج $ Solcore  $لتقييم خصائص $ QW $ أولاً حساب تأثير السلالة باستخدام $ 8-band Pikus-Bir Hamiltonian $ ، ومعالجة كل مادة في الهيكل على أنها كتلة ، ثم استخدام النطاقات المزاحة والكتل الفعالة لحل معادلة شرودينجر في بعد واحد  $ 1D Schödinger $ ، بعد المحاذاة الصحيحة بين الطبقات. أخيرًا ، يُحسب معامل الامتصاص بناءً على كثافة الحالات ثنائية الأبعاد ، بما في ذلك تأثير الإكسيتونات.

\subsubsection{Bulk 8-band kp solver}
هناك العديد من الطرق العددية لحساب بنية النطاق لمادة بدرجات متفاوتة من التطور العلمي  والتعقيد. مثل الجهد المستعار $ (pseudopotential) $  والربط القوي $ (tight banding ) $ أو دالة غرين $ (Green's function) $ أو طرق $ (K*p) $. يشتمل $ solcore  $ على $  8band Pikus-Bir Hamiltoian $ لحساب بنية النطاق للمواد السائبة تحت إجهاد ثنائي المحور، بالأخذ بعين الإعتبار التزاوج بين عصبة التوصيل، الفجوات الثقيلة، ضوء الفجوات والإنقسام في الطبقات.
ال $ eigenfunctions Ψ $ و $ eigenstates E$ هي حلول المعادلة التالية (في الشكل )، حيث $ H $ هو $ Pikus-Bir hamiltonian $:
********************figure2***********
تتكون المصطلحات الندبية من ثلاثة مكونات: المعلومات حول حواف النطاق غير المقيدة ، والمصطلح الحركي، ومصطلح الانفعال.
يتم حل هذا النظام بسهولة من أجل $ k_i $ المحدد باستخدام وظيفة $ linalg.eig $ الخاصة بـ $ Numpy $ ، والتي توفر وظائف $ eigenfunctions $ وقيم $ eigenvalues $ المقابلة. عادة ، نحن مهتمون بحواف الشريط الجديدة بسبب تأثير الإجهاد والكتل الفعالة الناتجة ، والتي يُعطى من خلال انحناء العصبات بالقرب من $ k_i = 0 $. يوضح (الشكل ) مثالاً للنطاقات المحسوبة بهذه الطريقة لحالة طبقة $ InGaAs $ المتوترة التي نمت بشكل زائف على $ GaAs $ ، والاعتماد الناتج للكتلة الفعالة مع محتوى الإنديوم للطبقة. لاحظ أنه بسبب تأثير الإجهاد ، لم تعد نطاقات الثقوب الثقيلة والخفيفة تتدهور عند نقطة جاما $ (K = 0) $.
------------------figure3-----------------
\subsubsection{معادلة شرودينجر في بعد واحد:}
بمجرد معرفة حواف النطاق الجديدة والكتل الفعالة لكل مادة من المواد التي تشكل بنية البئر الكمومي ، يمكن حساب خصائص الكم عن طريق حل معادلة شرودينجر أحادية البعد. يستخدم $ Solcore $ الطريقة التي وصفها $ Frensley $ ، والتي تسمح بحساب القيم الذاتية $ eigenvalues $ والمتجهات الذاتية للجهد الكيفي. ومع ذلك ، يتم تنفيذ شروط الحدود المغلقة والدورية فقط.
يتم إنشاء مصفوفة ثلاثية الأضلاع بكتابة معادلة شرودينجر ذات الكتلة الفعالة المتغيرة عبر سلسلة من نقاط الشبكة. تتوافق القيم الذاتية للمصفوفة مع مستويات الطاقة المسموح بها للنظام. وبالتالي ، يتم تقديم نظام المعادلات لحل نقاط الشبكة من خلال:
معادلة 5**********************
معادلة6/7*****************
يتم حل هذا النظام باستخدام الأدوات المتاحة في حزمة $ Scipy $ لحل أنظمة المعادلات الخطية المتفرقة ، ولا سيما $ sparse.linalg.eigs $.
still so much going on but I'm too lazy to continue 



\subsection{المصادر الضوئية:}
إن تحويل ضوء الشمس إلى كهرباء هو الهدف النهائي لأي خلية شمسية ، وبالتالي من الضروري أن يكون لديك طريقة ملائمة لإنشاء خصائص طيف الضوء ومعالجتها وتعديلها. من الناحية المثالية ، سيتم تصميم الخلايا الشمسية وتقييمها تحت طيف شمسي قياسي - على سبيل المثال الكتلة الهوائية 1.5 من الطيف الشمسي المباشر ، $ AM1.5D  $- لكن مصادر الضوء العملية ليست قياسية. في أغلب الأحيان ، مطورو $ solcore $ مهتمون بنمذجة أداء خلية شمسية تحت الطيف التجريبي لمحاكاة شمسية أو مصباح في المختبر ، بيانات محاكاة محسوبة من الظروف الجوية (درجة الحرارة ، الرطوبة ، محتوى الهباء الجوي ، إلخ) أو حتى أقل قياس بيانات الإشعاع الحقيقي في مواقع مختلفة حول العالم. يمكن بعد ذلك مقارنة ذلك بالأداء التجريبي أو تكييفه للعمل بشكل أفضل في ظل ظروف معينة.
تم تصميم مصدر ضوء وحدة $ Solcore $ للتعامل بسهولة مع مصادر الضوء المختلفة. لديه دعم مباشر لـ:
\begin{itemize}
	\item 
	
	الانبعاث الغاوسي ، النموذجي لليزر والصمامات الثنائية الباعثة للضوء.
	\item 
	إشعاع الجسم الأسود ، وهو سمة من سمات مصابيح الهالوجين التي تحددها درجة الحرارة ، ولكنها تستخدم أيضًا في كثير من الأحيان لمحاكاة طيف الشمس ، وهي قريبة جدًا من مصدر الجسم الأسود عند 5800 كلفن.
	\item 
	الأطياف الشمسية القياسية: الطيف خارج الأرض $ AM0 $ والطيف الأرضي $ AM1.5D $ و $ AM1.5G $ كما هو محدد في معيار $ ASTM G173-03 $ (2008).
	\item 
	نماذج الإشعاع ، باستخدام الموقع والوقت والمعلمات الجوية لحساب الطيف الشمسي الاصطناعي. يشتمل $ Solcore $ على نموذجين:
	$ SPECTRAL2 $ ، تم تنفيذه بالكامل في $ Python $ ،
	وواجهة لثنائيات $ SMARTS  $(التي يجب تثبيتها بشكل منفصل) ، مما يبسط استخدامه بشكل كبير في وضع الدُفعات.
	\item 
	الإشعاعات التي يحددها المستخدم ، والتي يتم توفيرها خارجيًا من قاعدة بيانات أو أي مصدر آخر ، مما يسمح بأقصى قدر من المرونة.
\end{itemize}
إن بناء الجملة في جميع الحالات بسيط وبديهي مع الأخذ في الاعتبار نوع المصدر الذي يجب إنشاؤه. في حالة نماذج الإشعاع ، التي غالبًا ما تحتوي على عدد كبير من المدخلات ، تحدد $ Solcore $ مجموعة من القيم الافتراضية ، لذلك يجب توفير القيم المختلفة فقط. يوضح الكود في القائمة 3 إنشاء العديد من مصادر الضوء باستخدام الحد الأدنى من المدخلات المطلوبة في كل حالة.

figure 4 *******************

\subsection{الحلول الضوئية:}
الغرض من هذه الحلول هو الحصول على إنعكاس الضوء الوارد وإمتصاصه ونقله في الخلية الشمسية كدالة للطول الموجي للضوء والموقع داخل الهيكل 
يتضمن $ Solcore $ ثلاثة نمادج لمعالجة هذه المشكلة :
\subsubsection{قانون بير -لامبيرت $ Beer–Lambert law (BL) $:}
هو أبسط نموذج لحساب الامتصاص في هيكل متعدد الطبقات. يتجاهل كل الإ نعكاس في الواجهات التي يمكن توفير انعكاس السطح الأمامي خارجيًا ، ويكون صفرًا 
يستخدم قانون $ BL $ على نطاق واسع في الخلايا الكهروضوئية ولكن في الواقع ، لا يمكن تطبيقه إلا عندما يمكن تجاهل التباين في معامل الانكسار بين الطبقات وعندما يكون هناك امتصاص قوي
الإمتصاص لكل وحدة طول كدالة لطول الموجة $ lambda\  $ والموضع $ z $ في الطبقة $ n $ يعطى بواسطة : 
\begin{equation}
	A_n(\lambda,z)=\alpha_n(\lambda)exp(-\sum_{i=1}^{ n-1}\alpha_i(\lambda)w_i-\alpha_ n ( \lambda )(z-z_n))
\end{equation}

حيث : $\alpha_n$ هي معامل إمتصاص الطبقة $ n $  ،سمكها  $ w_n $ ، موضع بداية الطبقة $  z_n $


\subsubsection{ طريقة مصفوفة النقل $ transfer matrix method (TMM) $ :}
من أجل تقييم  السلوك الضوئي لخلية شمسية من المهم مراعاة تفاعل الإشعاع الكهرومغناطيسي المتولد مع تعاقب كل من الطبقات المستوية  الممتصة و الغير الماصة ،$ Solcore $  يقوم بتقييم تفاعل الإشعاع الكهرومغناطيسي الساقط من خلال هيكل متعدد الطبقات بإستخدام $ TMM $ 
نستخدم نمودج $ TMM $  في $ Solcore $ عن طريق وحدة $ tmm $ التي طورها $ Byrnes $   

\subsubsection{التحليل الموجة المزدوجة الدقيق$  rigorous coupled-wave analysis (RCWA) $:}
يشتمل $ Solcore $ على واجهة $ S4 solver $ - التي يجب تتبيثها بشكل منفصل - والتي تم تطويرها في جامعة $ Stanford University $ ، من أجل تصميم الخلايا الشمسية النانوية . $ S4 $ هو تطبيق ل$ RCWA $ والتي يشار إليها أيضا في بعض الأحيان بإسم طريقة فورييه النموذجية $ FMM $  والتي تحل معادلات ماكسويل الخطية  في الهياكل التي تحتوي على دورية ثنائية الأبعاد 

يمكن حساب الضوء المنعكس والممتص والمرسل خارجيا تم توفيره كمدخل إلى $ Solcore $  للحصول على الخواص الكهربائية لهيكل الخلايا الشمسية.

\subsection{الخلايا الشمسية متعددة الوصلة:}
يمكن أن تشتمل الخلية الشمسية الكهروضوئية الكاملة على تقاطع واحد أو أكثر ، وملامسات معدنية ، وطبقات بصرية (بما في ذلك الطلاءات المضادة للانعكاس والتركيبات النانوية الضوئية) وتقاطعات الأنفاق. قد تتراوح التقاطعات ، بدورها ، من تماثلات $ PN $ البسيطة إلى الوصلات غير المتجانسة المعقدة ، بما في ذلك هياكل الآبار متعددة الكم. الحلول الموصوفة سابقا تحسب فقط خصائص أجهزة الوصلة الواحدة. لدمجها في جهاز متعدد الوصلات ، من الضروري مراعاة أن الوصلات الفردية متصلة كهربائيًا في سلسلة والاقتران المحتمل للضوء المنبعث من التقاطعات ذات فجوة الحزمة العريضة مع تلك ذات فجوة النطاق الأصغر. تتضمن المعلومات التكميلية مثالاً كاملاً خطوة بخطوة لنمذجة خلية شمسية مزدوجة الوصلات مع $ QWs  $وطلاء مضاد للانعكاس ووصلة نفق ، وحساب كفاءة الكم الخارجية ، وخصائص $ IV $ تحت الإضاءة وأداء الخلية الشمسية كدالة لتركيز الضوء.
\subsubsection{دون اقتران إشعاعي:}
نعتبر أولاً حالة عدم وجود اقتران إشعاعي بين الوصلات. هذا تقدير تقريبي جيد للمواد التي لا تشع بكفاءة أو المواد المشعة التي تعمل بتركيز منخفض ، عندما يكون جزء إعادة التركيب الإشعاعي مقارنة بإعادة التركيب غير الإشعاعي منخفضًا. في هذه الحالة ، يمكن حساب منحنى $ I V $ لكل وصلة بشكل مستقل عن بعضها البعض ويكون التيار المتدفق عبر هيكل $ MJ  $مقيدًا بالتقاطع مع أدنى تيار عند أي جهد معين. يتم الآن إضافة مقاومات التسلسل المحددة لكل تقاطع معًا وتضمينها كمصطلح واحد. يتم أخيرًا حساب جهد التشغيل لكل من التقاطعات وإضافته معًا للحصول على جهد الخلية $ MJ $.
\subsubsection{بوجود اقتران إشعاعي:}
يحدث الاقتران الإشعاعي عندما يتم امتصاص الضوء الناتج عن تقاطع ذو فجوة نطاق عالية بسبب إعادة التركيب الإشعاعي بواسطة وصلة ذات فجوة نطاق منخفضة ، مما يساهم في تياره الضوئي وتغيير نقطة التشغيل. تم التعرف عليه في العديد من المواد عالية الإشعاع مثل الخلايا الشمسية البئر الكمي والخلايا الشمسية $ III-V MJ $. يظهر كحقيقة أثناء قياسات $  $الخلايا الشمسية MJ ، ولكنه أيضًا تأثير يمكن استغلاله لزيادة أداء أجهزة $ MJ  $وتحملها للتغيرات الطيفية ، مما يؤدي إلى زيادة إنتاجية الطاقة المتحصل عليها
\subsubsection{تقاطع النفق:}
يتضمن $ Solcore $ دعمًا جزئيًا لتقاطعات النفق. إنها تمثل خسارة ضوئية بسبب الامتصاص المعيق في الطبقات ، ولكنها تمثل أيضًا خسارة كهربائية. يوجد حاليًا نموذجان لتقاطعات الأنفاق. الأول هو نموذج مقاوم بسيط ، حيث يتم تصميم تقاطع النفق ببساطة كمقاومة متسلسلة. يجب أن يكون هذا التقريب صالحًا في معظم الحالات ، ولكنه سينهار إذا كان التيار قريبًا من أو أعلى من كثافة تيار الذروة للتقاطع.
النموذج الثاني هو نموذج حدودي مبني على الشكلية البسيطة التي وصفها $ szi $. في هذا النموذج ، سيكون للتيار الإجمالي لتقاطع النفق ثلاثة مكونات: حساب $ JT $ الحالي للنفق للنقل من نطاق إلى نطاق ، والتيار الزائد $ Jex $ المرتبط بالنقل عبر الحالات داخل فجوة النطاق ، و $ JD $ الحالي للانتشار.

























