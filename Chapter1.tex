\chapter{Materiaux} % Main chapter title

\label{Chapter1} % Change X to a consecutive number; for referencing this chapter elsewhere, use \ref{ChapterX}

%----------------------------------------------------------------------------------------
%	SECTION 1
%----------------------------------------------------------------------------------------
\section*{مقدمة}


\chapter{عموميات على الخلية الشمسية}

\section {مفاهيم عن الخلية الشمسية  }
\subsection{نبذة عن تاريخ الخلايا الكهروضوئية:}
يعتقد الكثيرون أن الفضل في بدايات الخلايا الكهروضوئية يعود إلى العالم الفرنسي إدموند بيكريل الذي لاحظ أنه يمكن لبعض المواد أن تنتج طاقة عند تسليط الضوء عليها وتأكد أنه كمية الطاقة الكهربائية المولدة من بعض الأقطاب الكهربية الموضوعة في كهروليت تزداد عند تسليط الضوء على الأقطاب، أطلق على هذه الظاهرة لاحقا اسم التأثير الكهروضوئي.
في عام 1873 اكتشف ويلوغبي سميث أن عنصر السيليسيوم يتمتع بقدرة توصيل كهروضوئية وهذا الإكتشاف كان القاعدة التي بنى عليها ويليامس غريلس ادامز وريتشارد ايفرنزداي بحثهما الذي انتهى بتوصلهما لنتيجة أن السيلينيوم يقوم بتوليد الكهرباء عند تعرضه لأشعة الشمس 1876. تم تصنيع أول خلية كهروضوئية من وايفر السيلينيوم من قبل تشارلز فريتز 1883.
في عام 1953 دشنت مختبرات بيل مشروعا بحثيا لإنتاج أجهزة لتوفير مصدر طاقة، حيث اقترح العالم داريل تشابن استخدام خلايا شمسية وتم الموافقة على الفكرة. بدأ تشابن التجارب باستخدام الخلايا الضوئية المصنوعة من السيلينيوم لكن كفاءتها كانت $ 0.5~\%~   $ فقط، لاحقا انظم إليه العالمان كلفن فولر وجيرالد بيرسون اللذان كانا يعملان في مشروع لتطوير الترانزيستورات من السيليكون، وعملوا معا على تقنية السيليكون الناشئة لتصنيع الخلايا الشمسية. في عام 1954  انشئت أول خلية شمسية بكفاءة تصل إلى $ 5.7~\%~   $  تم تقديمها للناس \cite{a1}.
في البداية كان إنتاج الخلايا الشمسية السيليكونية مكلف جدا وكذلك كانت موجهة لتطبيقات الفضاء وطاقة الأقمار الصناعية، في نهاية السبعينات من القرن الماضي أنشئت صناعة الخلايا الشمسية الأرضية وبدأت في الانتشار بشكل واسع وكذلك بدأت تنخفض تكلفت إنتاجها.
\subsection{التأثير الكهروضوئي:}
عندما يصطدم فوتون ذو الطاقة $ E $ بالكترون موجود على سطح موصل معدني، يمكن أن يحدث إنبعاث للإلكترونات، تسمى هذه العملية بالتأثير الكهرو ضوئي. يتعلق انبعاث الإلكترونات بتردد الضوء الساقط وليس شدته، وقد قدم العالم اينشتاين تفسيرا لهذه الظاهرة بعد تفسير بلانك لظاهرة إشعاع الجسم الأسود.

\begin{figure}[h!]
	\centering
	\includegraphics[width=0.8\linewidth, height=0.4\textheight]{Fig/Fig_I/1}
	\caption{ظاهرة التأثير الكهروضوئي}
	\label{fig:1}
\end{figure}
\FloatBarrier

\subsection{طيف الإشعاع الشمسي}
يتكون الإشعاع الشمسي من طيف من موجات كهرومغناطيسية تقسم إلى نطاقات حسب أطوالها الموجية، ومن هذا الطيف الكبير للموجات الكهرومغناطيسية نشعر فقط بالموجات في نطاق الأطوال من 0.10 إلى 100 ميكرون (1 ميكرون هو 1 من مليون متر) حيث تسبب هذه الموجات إحساسنا بالحرارة وبالتالي تسمى بالإشعاع الحراري. ويحتوي الغلاف الجوي على غاز الأوزون وبخار الماء وجسيمات الهواء وبعض الجسيمات المعلقة كالغبار وقطرات الماء التي تؤدى كلها إلى إضعاف الإشعاع الشمسي نتيجة امتصاصه أو تبعثره في نطاقات موجية مختلفة لذلك لا يصل كل الإشعاع الشمسي إلى سطح الأرض. يمكن إعتيار الشمس جسسما أسودا عند درجة حرارة 5700 كلفن ويتم تمثيل الطيف الشمسي بالمنحنى التالي:
\begin{figure}[h!]
	\centering
	\includegraphics[width=0.8\linewidth, height=0.4\textheight]{Fig/Fig_I/2}
	\caption{طيف الإشعاع الشمسي}
	\label{fig:2}
\end{figure}
\FloatBarrier
\subsection{الخلايا الشمسية:}
الخلايا الشمسية هي وحدات البناء الأساسية للخلايا الكهروضوئية، تقوم على مبدأ تحويل الطاقة الكهروضوئية الشمسية إلى طاقة كهربائية، بحيث أنها عبارة عن شبه ناقل يتم معالجة سطحه لكي يعكس أقل قدر ممكن من الضوء ويطبع عليه نمط معين من معدن ناقل ليوصل الكهرباء، ولهذا فإن الفوتونات ذات التردد الواقع في الجزء المرئي لديها القابلية لإثارة إلكترونات الخلية. إن التيار الذي تنتجه الخلية مقبول نسبيا (يصل إلى بضع مئات ملليّ أمبير) لكنّ الجهد ضعيف جدا بالنسبة لمعظم التطبيقات $ (0.5v) $، لذا يتم توصيل حوالي 28 إلى 36 خلية في سلاسل لتولد جهدا من مضاعفات $ (12v) $ في ضروف الإضاءة القياسية، ثم تربط هذه السلاسل أحيانا في مصفوفات لتعطي جهد خرج وتيار أكبر (حسب الإستعمال). يتم ربط الخلايا مع دايودات المنع و ديودات المسار الجانبي لتجنب فقد الطاقة أو في حالة وجود خلل في إحدى الخلايا، كذلك يمكن إدراج وحدات تخزين لتنظيم الطاقة المنتجة بسبب التغير المستمر في الإضاءة.
\begin{figure}[h!]
	\centering
	\includegraphics[width=0.8\linewidth, height=0.4\textheight]{Fig/Fig_I/3}
	\caption{مخطط بسيط لنظام الخلية الشمسية}
	\label{fig:3}
\end{figure}
\FloatBarrier
تعمل الخلية الشمسية كدايود في الظلام بحيث في وجود جهد مطبق على الخلية في الظلام ينتج تيار مشابه لتيار الدايود يسمّى تيار الظلمة. أما تحت الإضاءة فتعمل الخلية عمل البطارية بحيث تولد قوة دافعة كهربائية $ emf $ مماثل للقوة الدافعة الكهربائية للبطارية، لكنها تختلف عنها في كون الطاقة المنتجة متفاوتة بحسب شدة الإضاءة، وأيضا يمكن إعتبار الخلية الشمسية مولّد تيار بدل مولد جهد.
\subsection{ أنواع الخلايا الشمسية: }
تتمثل بتصنيف الخلايا الشمسية إلى ثلاثة أجيال رئيسية:
\begin{itemize}
	\item الجيل الأول
	يمثل خلايا شرائح السليكون الشائعة الاستخدام بشكلها التقليدي وتحتل القطاع الأكبر في عالم صناعة الخلايا الكهروضوئية، وتتوفر بنوعين أحادي التبلور (مونو) ومتعددة التبلور (بولي) وتتميز الخلية الأحادية التبلور بأنها أعلى كفاءة من الخلية المتعددة التبلور.
	\item الجيل الثاني 
	يدعى بشرائح الأغشية الرقيقة وتتضمن السليكون الغير متبلور وتريليد الكادميوم $ (CdTe)  $وخلايا $ (CIGS) $ وتعتبر الأكثر فعالية من سابقتها في استخدامات مشاريع الطاقة الكبيرة وأنظمة المباني المتكاملة أو الأنظمة الصغيرة المستقلة.
	\item الجيل الثالث 
	يتضمن العديد من تقنية الأغشية الرقيقة (متعدد الوصلات) الحديثة النشأ والظهور ولا زالت في مرحلة البحث والتطوير ولم يتم إنتاجها بصورة تجارية.
\end{itemize}


\subsection{ضروف الإضاءة القياسية:}
تختبر كفاءة وقدرة خرج أي خلية شمسية في ظل الضروف القياسية التالية: شدة إضاءة مقدارها $ (1000W/m²) $، درجة حرارة محيطة $ (c°25) $، وطيف يرتبط بضوء الشمس الذي مرّ عبر الغلاف الجوي عندما كانت الشمس على إرتفاع $  (42°)  $ من الأفق حينما تكون كتلة الهواء $ (AM=1.5) $.
\subsection{واط الذروة:}
نسبة واط الذروة $ W_p $ لأي وحدة شمسية هي القدرة $ P $ المنتجة من قبل الوحدة في ظل الضروف القياسية عند نقطة القدرة القصوى.

\section{خصائص الخلية الشمسية:}
\subsection{تيار الدارة المستقصرة وفرق جهد الدارة المفتوحة:}
التيار المنتج من قبل الخلية الشمسية، تحت الإضاءة، عند ربط الأطراف أي دارة مستقصرة (لا يوجد مقاومة) ‫هو‬ ‫أقصى‬ ‫تيار‬ ‫تستطيع‬ ‫خلية‬ ‫شمسية‬ ‫انتاجه‬، يسمى تيار الدارة المستقصرة $ I_{sc} $.
\begin{equation}
	J_{sc}=q\  \int\ b_s  \left( E \right)\ QE \left( E \right)\ dE
	\label{معادلة 1}
\end{equation}

عند عزل الأطراف (أي مقاومة لا نهائية) يتولد جهد هو أقصى جهد تعطيه الخلية الشمسية، يسمى بجهد الدارة المفتوحة $ V_{oc} $.
\begin{equation}
	V_{oc}= \frac{KT}{q}\ \ln \left( \frac{J_{sc}}{J_0}\ +1 \right)
\end{equation}

\begin{figure}[h!]
	\centering
	\includegraphics[width=0.8\linewidth, height=0.4\textheight]{Fig/Fig_I/4}
	\caption{منحنى الجهد-التيار للخلية الشمسية}
	\label{fig:4}
\end{figure}
\FloatBarrier
\subsection{قدرة الخلية الشمسية وعامل الإمتلاء:} 
تتحدد قدرة الخرج للخلية بضرب التيار والجهد $ P=IV، $ تكون قدرة الخلية دائما أقل من $ I_{sc} $ و $ V_{oc} $، تصل القدرة إلى أعلى قيمة لها عند جهد وتيار يوافقان $ V_m $ و $ I_m $ على التوالي.
نعرف عامل الإمتلاء بالمعادلة:
\begin{equation}
	FF=\frac{J_m\ V_m}{J_{sc}\ V_{oc}}
\end{equation}
يتراوح عادة بين 0.80 و 0.85.
\subsection{معامل الكفاءة و الكفاءة الكمية:}
إنّ التيّار الكهروضوئي المولد من قبل الخلية الشمية -تحت الإضاءة- في دارة مستقصرة $ I_{sc} $ يعتمد على الطيف الساقط، لربط التيار المولد بالطيف الساقط نستعمل معامل الكفاءة الكمية $ QE $ ، حيث $ QE(E) $ هي إحتمالية تحرير فوتون ساقط ذو طاقة E لإلكترون إلى الدارة الخارجية، و $ bs(E)  $ هي كثافة التدفق الفوتون الطيفي الساقط كما هو موضح في 
\ref{معادلة 1}.
كفاءة الخلية هي كثافة القدرة المحررة عند نقطة التشغيل كجزء من كثافة الضوء الساقط $ P_s $،
\begin{equation}
	\eta=\frac{J_m\ V_m}{P_s}
\end{equation}
ويمكن ربطها مع $ I_{sc} $ و $ V_{oc} $ باستعمال $ FF $ كالتالي:
\begin{equation}
	\eta=\frac{J_{sc}\ V_{oc}\ FF}{P_s}
\end{equation}
\subsection{تيار الظلمة وسلوك الديود غير المثالي:}
في غياب مصدر الإضاءة (في الظلام) وفي وجود حمل، ينشأ فرق جهد بين أطراف الخلية، يولد هذا الفرق في الجهد تيارا يتدفق في الإتجاه العكسي للتيار الضوئي يسمّى تيار الظلمة. تسلك أغلب الخلايا الشمسية سلوك الديود المثالي حيث تسمح بمرور تيار عند جهد مباشر أكثر منه عند جهد عكسي، تمثل كثافة التيار في ديود مثالي بالمعادلة التالية:
\begin{equation}
	J_{dark}\left( V\right) = J_0\left( e^{\frac{qV}{K_bT}}-1\right) 
\end{equation}

يؤدي تيار الظلمة إلى خفض قيمة صافي التيار عن قيمته Isc، ويمكن تقريب قيمة صافِ التيّار بالجمع بين $ I_{sc} $ و $ I_{dark} $ ، تسمى هذه العملية بالتراكب المقرّب، يعطى بالعلاقة

\begin{equation}
	J= J_{sc} - J_0\left( e^{\frac{qV}{K_bT}}-1\right) 
\end{equation}
في الحقيقة نادرا ما نرى سلوك الديود المثالي، لكن من الشائع أن تيار الظلمة لا يعتمد كثيرا على الجهد، بل يتم تحديد الإعتماد الفعلي على V بواسطة معامل المثالية m (حيث يتراوح بين 1 و 2 في ديود غير  مثالي) 
\begin{equation}
	J= J_{sc} - J_0\left( e^{m \frac{qV}{K_bT}}-1\right) 
\end{equation}
\subsection{الدارة المكافئة للخلية الشمسية:}
لدراسة الخلية الشمسية كهربائيا يمكن نمذجتها بدارة كهربائية ذات عناصر معلومة الخصائص والأداء، بحيث تمثل بمولد تيار مربوط على التوازي مع دايود ومقاومتين إحداهما على التسلسل والأخرى على التوازي كالتالي: 
\begin{figure}[h!]
	\centering
	\includegraphics[width=0.8\linewidth, height=0.4\textheight]{Fig/Fig_I/5}
	\caption{الدارة الكهربائية المكافئة للخلية الشمسية}
	\label{fig:5}
\end{figure}
\FloatBarrier
في الخلايا الحقيقية ، تتبدد الطاقة من خلال مقاومة العناصر ومن خلال التيار المتسرب حول جوانب الجهاز. هذه التأثيرات مكافئة كهربائيًا للمقاومتين على التسلسل $ (R_s) $ وعلى التوازي$ (R_{sh}) $ مع الخلية. بحيث كلّما زادت$  R_s  $أو نقصت$ R_{sh} $تنخفض قيمة $ P  $(وبالتالي تنقص من قيمة $ FF $ ) أي أنه في خلية مثالية تكون $ R_s=0  $و  $ R_{sh}=infini $
\begin{figure}[h!]
	\centering
	\includegraphics[width=0.8\linewidth, height=0.4\textheight]{Fig/Fig_I/6}
	\caption{تأثير $ R_s $و $ R_sh $ على منحنى الجهد والتيار للخلية الشمسية}
	\label{fig:fig6}
\end{figure}
\FloatBarrier
\section{مبدأالتوازن التفصيلي:}
ينشأ أحد القيود الفيزيائية الأساسية على أداء الخلية الكهروضوئية من مبدأ التوازن المفصل. بالإضافة إلى امتصاص الإشعاع الشمسي ، يقوم محول الطاقة الشمسية بتبادل الإشعاع الحراري مع محيطه. تشع كل من الخلية والبيئة المحيطة بها فوتونات ذات طول موجي طويل وحراري بسبب درجة حرارتها المحدودة. يجب أن يتطابق معدل انبعاث الفوتونات من الخلية مع معدل امتصاص الفوتون ، بحيث يظل تركيز الإلكترونات في المادة ثابتًا في الحالة المستقرة. 
تم تقديم حد التوازن التفصيلي من قبل $   Shockley $ و $ Queisser  $في عام 1961 و يمثل الحد الأقصى للكفاءة النظرية لخلية شمسية باستخدام تقاطع $  p-n $ منفرد لجمع الطاقة من الخلية حيث آلية الخسارة الوحيدة هي إعادة التركيب الإشعاعي في الخلية الشمسية والتي تحدث نتيجة اعادة اتحاد الالكترونات بالفجوات وتسمى ظاهرة اعادة التركيب $ recombination $ مما يخفض من كمية الالكترونات الحرة التي ستحدث تيار كهربائي نتيجة لضياع طاقة الفوتونات الضوئية في اعادة تحرير الالكترونات واطلاقها مرة ثانية.

\section{نظرة عن أشباه النواقل}

\subsection{أشباه النواقل:}
عندما يتم تجميع زوج من الذرات معًا في جزيء ، تتحد مداراتها الذرية لتكوين أزواج من المدارات الجزيئية مرتبة أعلى قليلاً وأقل قليلاً في الطاقة من كل مستوى أصلي. نقول أن مستويات الطاقة قد انقسمت. عندما يجتمع عدد كبير جدًا من الذرات معًا في مادة صلبة ، ينقسم كل مدار ذري إلى عدد كبير جدًا من المستويات ، قريبة جدًا من بعضها البعض في الطاقة بحيث تشكل بشكل فعال سلسلة متصلة أو نطاقًا من المستويات المسموح بها.يعتمد توزيع الطاقة في العصابات (النطاقات) على الخصائص الإلكترونية للذرات وقوة الترابط بينها. تكون النطاقات مشغولة إذا كانت المدارات الجزئية الأصلية مشغولة (والعكس صحيح). عادةً ما يُطلق على أعلى نطاق مشغول ، والذي يحتوي على إلكترونات التكافؤ ، نطاق التكافؤ $ (VB) $. أدنى نطاق غير مشغول يسمى نطاق التوصيل $ (CB) $. إذا كان شريط التكافؤ ممتلئًا جزئيًا، أو إذا كان يتداخل في الطاقة مع أدنى نطاق غير مشغول ، فإن المادة الصلبة عبارة عن معدن. إذا كان شريط التكافؤ ممتلئًا تمامًا ومنفصلًا عن النطاق التالي بفجوة طاقة، فإن المادة الصلبة تكون عبارة عن شبه موصل أو عازل. تشترك جميع الإلكترونات في نطاق التكافؤ في الترابط ولا يمكن إزالتها بسهولة. تتطلب طاقة مكافئة لفجوة النطاق لإزالتها إلى أقرب مستوى متاح غير مشغول. لذلك فإن هذه المواد لا توصل الحرارة أو الكهرباء بسهولة.
\\تختلف أشباه النواقل عن العوازل في فجوة النطاق، فإذا كانت $ Eg<0.5ev  $ تسمى شبه معدن وإذا كانت $ Eg<3ev  $ تسمى عازلا، أما إذا كانت بين $ 0.5ev $ و $ 3ev $ فهي شبه ناقل. تتمتع أشباه الموصلات بموصلية صغيرة في الظلام لأن عددًا صغيرًا فقط من إلكترونات التكافؤ سيكون لديها طاقة حركية كافية في درجة حرارة الغرفة ليتم تحفيزها عبر فجوة النطاق في درجة حرارة الغرفة. تتناقص هذه الموصلية الذاتية مع زيادة فجوة النطاق.
\begin{figure}[h!]
	\centering
	\includegraphics[width=0.8\linewidth, height=0.4\textheight]{Fig/Fig_I/7}
	\caption{العصبات في النواقل، أشباه النواقل والعوازل}
	\label{fig:7}
\end{figure}
\FloatBarrier
\subsection{بنية العصبات:}
لتحديد مستويات الطاقة لذرة أو جزيء، نحتاج إلى حل معادلة شرودنجر، ولكن يجب أن نأخذ في الإعتبار أن هناك مجموعة لانهائية من الجهود الذرية في المادة البلورية، يمكننا الإستفادة من كونها دورية لانهائية في كون التوزيع الإحتمالي للإلكترونات يجب أن يكون دوريا أيضا مستغلة تناظرها. نظرًا لأن الشبكة غير محدودة ، يجب أن تشكل الإلكترونات حالات غير محددة تمتد عبر البلورة ، تمامًا مثل الإلكترون في الفضاء الحر. ولذا فإن الدالة الموجية التي تستوفي الشروط هي:   
\begin{equation}
	\psi \left( K,r \right)= U_{ik}\left( r \right) e^{ik\ r} 
\end{equation}
حيث $ k $ يسمى $ wavevector $. لكل قيمة $ k $ ، توجد حلول متعددة لمعادلة شرودنغر المسمى $ n $ ، مؤشر النطاق، الذي يقوم ببساطة بترقيم نطاقات الطاقة. يتطور كل مستوى من مستويات الطاقة هذه بسلاسة مع التغييرات في $ k $ ، مما يشكل نطاقًا سلسًا من الحالات. لكل نطاق يمكننا تحديد دالة $ E_n(k) $ ، وهي علاقة التشتت للإلكترونات في هذا النطاق.
\\
يأخذالدالة الموجية أي قيمة داخل منطقة بريليون ، وهي عبارة عن متعدد الوجوه في مساحة الموجة (شعرية متبادلة ) مرتبطة بشبكة البلورة. تتوافق موجات الموجات خارج منطقة بريليون ببساطة مع الحالات المتطابقة فعليًا مع تلك الحالات داخل منطقة بريليون. من الشائع في الأدبيات العلمية أن نرى مخططات بنية النطاق والتي تُظهر قيم $ E_n(k) $ لقيم $ k $ على طول الخطوط المستقيمة التي تربط نقاط التماثل ، وغالبًا ما يتم تسميتها$  Δ $  $ Λ $  $ Σ $ أو $ [100] $ و $ [111] $ و $ [110] $ على التوالي. هناك طريقة أخرى لتصور بنية النطاق وهي رسم شكل ثابت للطاقة متساوي السطح في فضاء الموجة ، مع إظهار جميع الحالات ذات الطاقة المساوية لقيمة معينة. يُعرف السطح المتساوي للحالات ذات الطاقة التي تساوي مستوى فيرمي باسم سطح فيرمي.
\subsubsection{عصبة التوصيل:}
في عصبة التوصيل، أدنى طاقة $ E(K) $ تكون عند النقطة $ K=0 $ (أو  قيمة أخرى لـ $ k $ تقابل اتجاهًا مهمًا في البلورة). عند أدنى طاقة بالقرب من النقطة $ K=K_{0c} $، يكون من المناسب توسيع $ E(K) $، تختصر بالمعادلة:
\begin{equation}
	E\left( K\right) = E_{c0} + \dfrac{ 
		\hbar^{2}\left| K-K_{0c}\right| }{2m^{*}_c}
\end{equation}

حيث $ E_{c0}=E_c(K_0) $ و $ m^{*}_c $ هي الكتلة الفعالة في عصبة التوصيل، تكون مشابهة لكتلة لإلكترون حر $ m_0 $ لكنها تختلف حسب القوة المطبقة على الإلكترون في الشبكة البلورية، بحيث إذا كانت $ m^{*}_c $ كبيرة يدل ذلك على أن الكترونات التوصيل متؤثرة بجهد ذري. $ m^{*}_c $ تصف عزم إلكترونات التوصيل $ P $ عند تطبيق قوة $ F  $
\begin{equation}
	F= m^{*}_c\ \frac{dp}{dt}
\end{equation}
\begin{equation}
	p= m^{*}_c\ V= \hbar \left( K-K_{0c}\right) 
\end{equation}
\subsubsection{عصبة التكافؤ:}
في عصبة التكافؤ نعبر عن طاقة الفجوات بالقرب من الحد الأقصى لعصبة التكافؤ، بحيث أن الفجوات تمثل غياب الإلكترونات. تكون الفجوات أكثر إستقرارا عندما تكون للإلكترونات طاقة قصوى، لذا فمن المرجح أن الثقوب تتواجد بالقرب من الحد الأقصى لنطاق التكافؤ، تكون لها طاقة حركية تزداد مع انخفاض $ E(K) $. عند $ K=K_0 $، يمكن تقريب الطاقة كالتالي:
\begin{equation}
	E\left( K \right) = E_{v0} + \dfrac{ 
		\hbar^{2} \left| K-K_{v0} \right| }{2m^{*}_v}
\end{equation}

حيث $ m^{*}_v $ هي الكتلة الفعالة في عصبة التكافؤ، تكون عادة موجبة. أما العزم فيعطى بالمعادلة:
\begin{equation}
	p= - \hbar \left( K-K_{v0} \right) 
\end{equation}
\\
عموما الكتلة الفعالة للإلكترونات والفجوات في شبه ناقل واحد تكون مختلفة بسبب اختلاف الإنحناء في عصبة التوصيل وعصبة التكافؤ.

\subsubsection{فجوة النطاق:}

تسمى أدنى قيمة طاقة قادرة على تحرير إلكترون بفجوة النطاق الأساسي $ Eg $. إذا كانت أدنى مستوى في عصبة التوصيل وأعلى مستوى في عصبة التكافؤ يوافقان نفس النقطة $ K $، حينها يكون للفوتون القابلية على انتاج زوج الكترون-فجوة. يسمى هذا النوع من أشباه الموصلات بالمواد ذات فجوة نطاق مباشرة.
إذا كانت أدنى مستوى في عصبة التوصيل وأعلى مستوى في عصبة التكافؤ عند نقاط مختلفة ل $ K $، فإن فوتون ذو طاقة $ Eg $ ليس لديه القدرة على إنتاج زوج إلكترون-فجوة، لأن نقل إلكترون من عصبة التكافؤ إلى عصبة التوصيل يتسبب في تغيير عزم الإلكترون، وبما أن العزم محفوظ في البلورة فإن الإلكترون يحتاج إلى تعويض الفرق في العزم $ n(k_{c0}-k_{v0})  $، يمتد الإلكترون هذا العزم من الفونون (إهتزاز الشبيكة البلورية). يسمى هذا النوع بشبه موصل ذو فجوة نطاق غير مباشرة. بإختصار في هذا النوع من أشباه الموصلات يتم إمتصاص الفوتون فقط إذا وجد الكم الكافي من الفونونات.
\begin{figure}[h!]
	\centering
	\includegraphics[width=0.8\linewidth, height=0.4\textheight]{Fig/Fig_I/8}
	\caption{فجوة نطاق مباشرة وغير مباشرة}
	\label{fig:8}
\end{figure}
\FloatBarrier
\subsection{المواد شبه ناقلة:}
تنتسب إلى أشباه الموصلات فئة كبيرة من المواد المختلفة وتنقسم إلى:
\begin{itemize}
	\item مواد بسيطة: وهي عناصر من الجدول الدوري، أشهرها الجرمانيوم $ Ge $ والسيليكون $ Si $ الذي يمثل ثاني أكثر عنصر متواجد على الأرض.
	\item مواد مركبة: يتم فيها الجمع بين عناصر الجدول الدوري للتحكم في خصائص شبه الموصل، مثل عناصر العمود الرابع معا $ SiC $، عناصر العمود الرابع والسادس $ Pbs $، عناصر العمود الثاني والسادس $ ZnO $، وعناصر العمود الثالث والخامس $ GaAs $ $ InP $....
	\\
	أشباه الموصلات العمود $ III-V $ ، مثل $ GaAs $ و $ InP $ ، هي أشباه موصلات ذات فجوة نطاق واقعة في الجزء المرئي، بالقرب من الأشعة تحت الحمراء، تتميز هذه المواد بحركة إلكترونات عالية جدا وذلك بسبب أن الكتلة الفعالة صغيرة جدا.
	
\end{itemize}
\subsection{أشباه النواقل المشوبة:}
في البلورة المثالية مستويات الطاقة المسموحة هي مستويات $ K $ المعرفة فقط، تعرف خصائص هذه المستويات بموقع مستوى فيرمي في وضع التوازن، وكذلك تعتمد على الطاقة الحرارية لتوصيل الطاقة الكهربائية.
\\
لكن إذا تم تشويب هذه البلورة ستنتج روابط بطاقات مختلفة، وبالتالي يختلف توزيع مستويات الطاقة الإلكترونية (إذا كانت مستويات الطاقة للشوائب تتداخل مع فجوة النطاق للبلورة يمكن أن تؤثر على الخصائص الإلكترونية لشبه الناقل). بإختصار فإن عملية الإشابة تزيد نسبة الإلكترونات إلى الفجوات -أو العكس- في وضع التوازن، وبالتالي يمكن التحكم في كثافة وطبيعة حاملات الشحنة في أشباه النواقل عن طريق إضافة كميات محددة من الشوائب (تصل إلى واحد من مليون)  بمستويات طاقة قريبة من حافة نطاق التوصيل أو التكافؤ. وهناك نوعان :
\begin{itemize}
	\item تشويب نوع $ N $: 
	تسمى أشباه الموصلات التي تم تشويبها لزيادة كثافة الإلكترونات بالنسبة للفجوات بالنوع $ n $(حاملات الشحنة الرئيسية سالبة). يتم  استبدال بعض الذرات في الشبكة البلورية بذرات الشوائب التي تمتلك عددًا كبيرًا جدًا من إلكترونات التكافؤ بالنسبة لعدد الروابط البلورية، تسمى بالذرات المانحة. مثال على ذلك ذرة الفوسفور حيث تحتوي على 5 إلكترونات تكافؤ عند إضافتها إلى بلورة سيليكون (رباعي التكافؤ) يبقى إلكترون حر وبالتالي فإن فصل هذا الالكترون عن الذرة لا يحتاج إلى طاقة كبيرة . 
	
	\item تشويب نوع $ P $: 
	تسمى أشباه الموصلات التي يتم تشويبها لزيادة كثافة الفجوات بالنسبة إلى الإلكترونات بالنوع $ p $. يتم إنتاجه عن طريق استبدال بعض الذرات في البلورة بذرات الشوائب المستقبلة ، والتي تساهم في عدد قليل جدًا من إلكترونات التكافؤ بالنسبة للروابط التي يحتاجون إليها لتكوينها. مثال على ذلك عنصر البورون ثلاثي التكافؤ في بلورة  السيليكون. يتأين المستقبل عن طريق إزالة إلكترون التكافؤ من رابطة أخرى لإكمال الترابط بينه وبين جيرانه الأربعة، مما يخلق ثقب في عصبة التكافؤ
\end{itemize}
	\begin{figure}[h!]
		\centering
		\includegraphics[width=0.8\linewidth, height=0.4\textheight]{Fig/Fig_I/9}
		\caption{شبه ناقل نوع $ n $ ونوع $ p $}
		\label{fig:9}
	\end{figure}
	\FloatBarrier
\section{كثافة الحالة:}
	حسب مبدأ الإستبعاد لباولي فإن كل حالة كمية تحمل إلكترونين بلف مختلف. وبما أن كل حالة كمية في البلورة تعرف ب$ K $ مختلف، يجب أن يوجد إلكترونين لكل قيمة $ K $. 
	من الأفضل تعريف كثافة الحالة في كل نطاق بدلالة الطاقة $ E(K) $ .
	كثافة الحالة في عصبة التوصيل : 
	\begin{equation}
		g_c\left( E \right) = \frac{1}{2 \Pi^{2}}\ \left( \dfrac{2m^{*}_c}{\hbar^{2}} \right)^{\frac{3}{2}}\left(  E_{c0} - E \right)^{\frac{3}{2}}   
	\end{equation}
	كثافة الحالة في عصبة التكافئ : 
	\begin{equation}
		g_v\left( E \right) = \frac{1}{2 \Pi^{2}}\ \left( \dfrac{2m^{*}_v}{\hbar^{2}} \right)^{\frac{3}{2}}\left(  E_{v0} - E \right)^{\frac{3}{2}}   
	\end{equation}
	في وضع التوازن أي عند درجة حرارة 0 كلفن تكون عصبة التكافؤ مملوءة تماما وعصبة التوصيل فارغة تماما، حيث يكون مستوى فيرمي في مكان ما في فجوة النطاق. نستعمل اختصار بولتزمان لتبسيط التكامل فنحصل على كثافة الحالة في عصبة التوصيل هي: 
	\begin{equation}
		N_c = 2 \left( \dfrac{m^{*}_c\ K_b T}{2 \pi \hbar^{2}} \right) ^{\frac{3}{2}}
	\end{equation}
	وفي عصبة التكافؤ هي :
	\begin{equation}
		N_v = 2 \left( \dfrac{m^{*}_v\ K_b T}{2 \pi \hbar^{2}} \right) ^{\frac{3}{2}}
	\end{equation}
	أما كثافة الحالة عند تطبيق جهد فهي كالتالي:
	كثافة الإلكترونات 
	\begin{equation}
		n= N_c\ e^{-\dfrac{\left( E_c-E_{Fn} \right) }{K_B\ T_n}}
	\end{equation}
	كثافة الفجوات
	\begin{equation}
		p= N_v\ e^{-\dfrac{\left( E_{Fp}-E_v \right) }{K_B\ T_p}}
	\end{equation}
	حيث $ T_n $ و $ T_p  $ هي الحرارة الفعالة للإلكترونات والفجوات على التوالي، يمكن أن تكون مختلفة عن حرارة المحيط
	
\section{ تيار الإلكترونات والفجوات:}
	
	يتم تمثيل كثافة الإلكترون والثقوب بالتوزيعات الاحتمالية للموجات المستوية. لإيجاد تيار الإلكترونات والفجوات، نحتاج إلى وزن احتمالية كل كثافة حاملات الشحنة بواسطة سرعة مجموعتها. بالتماثل مع نتيجة الإلكترونات الحرة ، فإن سرعة المجموعة للإلكترون ذي الكتلة الفعالة $ m^{*}_c $ في الحالة $ k $ هي $ \frac{\eta\ k}{ m^{*}_c} $. كثافة تيار الإلكترون الصافي في نطاق التوصيل هي:
	معادلة
	\begin{equation}
		n\left( r \right)= \int_{CBK} g_c\left( K\right)\ f \left( K,r \right)d^{3}K 
	\end{equation}
	تيار الفجوات في نطاق التوصيل هو: 
	\begin{equation}
		n\left( r \right)= \int_{VBK} g_v\left( K\right)\left( 1- f \left( K,r \right)\right) d^{3}K 
	\end{equation}
	في وضع التوازن يكون تيار الإلكترونات والفجوات معدوم أي أنه لا يوجد تدفق صاف للتيار في شبه ناقل عند وضع التوازن.
	اما عند تطبيق جهد فيعطى تيار الإلكترونات والفجوات عند أي نقطة: 
	معادلة 
	\begin{equation}
		J_n\left( r \right)= \mu_{n}\ n \nabla E_{Fn} 
	\end{equation}
	\begin{equation}
		J_p\left( r \right)= \mu_{p}\ n \nabla E_{Fp} 
	\end{equation}
	تعطى حركية الإلكترونات والفجوات ب 
	\begin{equation}
		\mu_{n}= \dfrac{qD_{n}}{K_BT}
	\end{equation}
	\begin{equation}
		\mu_{p}= \dfrac{qD_{p}}{K_BT}
	\end{equation}
	عند أي نقطة r يعطى صاف التيار بمجموع تيار الإلكترونات والفجوات بالمعادلة
	\begin{equation}
		J\left( r \right)= J_{n}\left( r \right)+J_{p}\left( r \right) 
	\end{equation}
	
\section{ التوليد وإعادة التركيب:}
	التوليد هو عملية إثارة إلكترونية تؤدي إلى زيادة ناقلات الشحنة، تحتاج هذه العملية إلى طاقة، يمكن توفيرها من إهتزاز الشبيكة البلورية (الفونونات) أو الضوء (الفوتونات) أو من الطاقة الحركية لحامل شحنة آخر. أما إعادة التركيب فهو عملية الإسترخاء الإلكتروني مما يؤدي إلى خفض عدد حاملات الشحنة، تنتج عن هذه العملية طاقة (بالآلية نفسها لعملية التوليد). لكلّ عملية توليد عملية إعادة تركيب مكافئة.
	
	\begin{itemize}
		\item 
		يمكن أن تكون عملية التوليد تحرير إلكترون من عصبة التكافؤ إلى عصبة التوصيل مما يخلق زوج فجوة-إلكترون، أو من موضعه في عصبة التكافؤ إلى حالة موضعية في فجوة النطاق مما يولد فجوة فقط، أو من حالة موضعية في فجوة النطاق إلى عصبة التوصيل مما يخلق إلكترون فقط. بالنسبة للخلايا الشمية أهم شكل من أشكال التوليد هو التوليد البصري الذي يعتمد على إمتصاص الفوتونات.
		\item إعادة التركيب هو فقدان الإلكترون أو الثقب من خلال اضمحلال الإلكترون إلى حالة طاقة أقل. قد يكون هذا من عصبة إلى عصبة ، مما يؤدي إلى إختفاء زوج فجوة-إلكترون ، أو قد يكون من نطاق التوصيل إلى حالة موضعية أو من حالة موضعية إلى نطاق التكافؤ ، مما يؤدي إلى إختفاء الإلكترون أو الفجوة فقط ، على التوالي. يمكن أن تكون الطاقة المنبعثة فوتون (إعادة التركيب الإشعاعي) أو فونون (إعادة التركيب غير الإشعاعي) أو كطاقة حركية إلى حامل شحنة آخر (إعادة تركيب أوجيه).
		
		
	\end{itemize}
	
	\section{وصلات $ PN $ في الخلايا الشمسية}
	\subsection{الوصلة $ pn $:} 
	التقاطع $ pn $ هو النموذج الكلاسيكي للخلايا الشمسية، يتم إنشاء هذه الوصلة بتشويب منطقتين من شبه ناقل بشكل مختلف أي يصبح لدينا واجهة من طبقة $ p $ و طبقة لنفس المادة. وبما أن دالة الشغل في $ p $ أكبر من $ n $ فإن الجهد الإلكتروستاتيكي في الجزء أصغر $ n $ من الجزء $ p $ وبالتالي يتولد حقل كهربائي في الوصلة يقود الإلكترونات المولدة نحو الجوء $ n $ و الفجوات نحو الجزء $ p $ .منطقة الوصلة (أي المنطقة بين $ p $ و $ n $ ) تكون دائما مستنفذة من الإلكترونات والفجوات وتمثل حاجز لناقلات الشحنة الأعضمية، ومسار منخفض المقاومة لحاملات الشحنة الأقلية. أي أنه يقود مجموعة ناقلات الشحنة الأقليات المولدة ضوئيًا عبر طبقات $ p $ و $ n $ ، وتصل إلى الوصلة عن طريق الانتشار.
	\subsection{الوصلة p-i-n:}
	الوصلة $ p-i-n $هو الإختلاف في السمة في $ pn $ . أي أنه وصلة $ pn $ بحيث تترك طبقة من شبه الناقل بين $ p $و $ n $بدون تشويب (تسمى بالطبقة الداخلية ). ينشأ نفس الجهد كما هو الحال في الوصلة $ pn $ لنفس مستويات التشويب، لكن الحقل الكهربائي يمتد عبر منطقة أوسع. يفضل هذا التصميم في المواد التي تكون فيها حاملات الشحنة الأقلية لهاأطوال إنتشار قصيرة ومن غير المرجح أن تساهم ناقلات الشحنة المولدة ضوئيا في الجزء $ p $ أو $ n $ في التيار الضوئي. تقاد حاملات الشحنة المولدة ضوئيا في المنطقة $ i $ نحو العناصر الكهربائية المربوطة مع الوصلة (الدارة الخارجية) بواسطة الحقل الكهربائي لتقطع مسافة أكبر منها في المواد المشوبة، كما أن عمر حاملات الشحنة في المنطقة $ i $ يكون أكبر من عمرها في الأجزاء المشوبة. تتمثل عيوب الوصلة $ p-i-n $ في أن الموصلية في المنطقة أضعف منها في المناطق $ p $ و $ n $ ، كما أنها تخلق مقاومة متسلسلة $ Rs $ . كذلك فإن احتمال إعادة التركيب داخل الطبقة الأولى في ظروف التحيز الأمامي ، حيث تتماثل نسبة الإلكترونات والفجوات. وأيضا قد تتسبب الشوائب المشحونة في انخفاض المجال الكهربائي إلى الصفر داخل المنطقة الجوهرية $ i $.
	\subsection{الوصلات غير المتجانسة: }
	يمكن أيضًا تحضير الوصلات $ p n $ و $ p-i-n $ كوصلات غير متجانسة باستخدام مادتين مختلفتين ذات فجوات نطاق مختلفة، يتم تحضير هذا النوع من الوصلات إما لرفع كمية حاملات الشحنة المجمعة أو للضرورة بسبب خصائص المواد المشوِّبة (أي المضافة إلى شبه الناقل) المتاحة. عند التقاطع يتشكل انقطاع في حواف نطاق التوصيل والتكافؤ بسبب التغيير في فجوة النطاق. تؤدي هذه الخطوة إلى مجالات فعالة مختلفة للإلكترونات والفجوات التي تكون عادةً مع المجال الكهروستاتيكي لحامل شحنة واحد وعكسه مع الآخر.
\subsection{الخلايا الشمسية متعددة الوصلة:}
	الخلايا الشمسية متعددة الوصلات $ (multi junction ) $ هي خلايا شمسية ذات تقاطعات $ p-n $ متعددة مصنوعة من مواد أشباه موصلات مختلفة . سينتج تقاطع $ pn $ لكل مادة تيارًا كهربائيًا استجابة لأطوال موجية مختلفة من الضوء . يسمح استخدام مواد شبه موصلة متعددة بامتصاص نطاق أوسع من الأطوال الموجية ، مما يحسن ضوء الشمس للخلية إلى كفاءة تحويل الطاقة الكهربائية.
	\\يمكن للخلايا المصنوعة من طبقات مواد متعددة أن يكون لها فجوات نطاقية متعددة ، وبالتالي ستستجيب لأطوال موجات ضوئية متعددة ،
	على سبيل المثال ، إذا كان لدى أحدهم خلية بها فجوات نطاق ، واحدة مضبوطة على الضوء الأحمر والأخرى إلى اللون الأخضر ، فإن الطاقة الإضافية في الضوء الأخضر والأزرق السماوي ستضيع فقط في فجوة النطاق للمادة الحساسة للأخضر ، في حين أن طاقة اللون الأحمر والأصفر والبرتقالي ستفقد فقط بسبب فجوة الحزمة الخاصة بالمادة الحساسة للأحمر. بعد تحليل مشابه لتلك التي تم إجراؤها للأجهزة ذات فجوة النطاق الواحدة ، يمكن إثبات أن فجوات النطاق المثالية لجهاز ثنائي الفتحات تبلغ 0.77  فولت و 1.70  فولت. بشكل ملائم ، لا يتفاعل الضوء ذو الطول الموجي المعين بقوة مع المواد ذات فجوة النطاق الأكبر. هذا يعني أنه يمكنك إنشاء خلية متعددة الوصلات عن طريق وضع طبقات من المواد المختلفة فوق بعضها البعض ، وأقصر أطوال موجية (أكبر فجوة نطاق) على "القمة" وزيادة خلال جسم الخلية. نظرًا لأن الفوتونات يجب أن تمر عبر الخلية للوصول إلى الطبقة المناسبة المراد امتصاصها ، يجب استخدام الموصلات الشفافة لتجميع الإلكترونات المتولدة في كل طبقة.
\subsubsection{تطورها}
\begin{itemize}
\item 

ابتكرت الخلايا الشمسية متعددة الوصلات أول مرة واستخدمت لإمداد الأقمار الصناعية بالطاقة حيث عوضت كفاءتها العالية المصروفات الباهظة لتحضيرها .
\item
وتستخدم هذه الخلايا حاليا أيضا على الأرض في مجمعات تعمل بالتاثير الضوئي الجهدي ، وقد أدى الجمع بين الكفاءة العالية والتركيز في الاستخدام إلى منافستها للألواح الشمسية التي تعمل بالسيليكون .
\item
وتستخدم تلك التقنية حاليا أيضا على مسباري المريخ أبورتيونيتي وسبيريت الذان أرسلا إلى المريخ عام $ 2004 $ ولا زالا يعملان حتى الآن .
\item
ويقدم الجمع بين خلايا ضوئية موصلة على التوالي ومكونة من فوسفيد الجاليوم إنديوم $ GaInP $ و أرسينيد الجاليوم $ GaAs $ و وصلات بي إن للجرمانيوم $ p-n junctions $ حلولا جيدة للاستخدام وبدأ الطلب يزداد عليها . وفي خلال الفترة بين ديسمبر $ 2006 $ وديسمبر $ 2007 $ زاد سعر الجاليوم عالى النقاوة من $ 350 $ دولار إلى $ 680 $ دولار للكيلوجرام . كما ارتفع حاليا ثمن معدن الجرمانيوم أيضا من $ 1000 $ دولار للكيلوجرام إلى $ 1200  $دولار . تلك المواد تستخدم لتمنية البلورات اللازمة لتلك الخلايا .
\item
وقد استخدم نوع من الخلايا الشمسية - جاليوم أرسينيد ثلاثي الوصلات - في تسير السيارات خلال السباق العالمي «ورلد سولار تشالنج» لاستغلال الشمس وكسبت الجائزة الأولى أربعة مرات متتالية من $ 2005 $ إلى $ 2007 $ .
\item
وفي عام $ 2009 $ اعلنت احدى الشركات أنها رفعت من كفاءة الخلية الضوئية الثلاثية الوصلات إلى $ 35 ~\%~ $ بزيادة $ 15 ~\%~ $ عن سابقتها .
\end{itemize} 
